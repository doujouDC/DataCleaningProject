\documentclass[]{article}
\usepackage{lmodern}
\usepackage{amssymb,amsmath}
\usepackage{ifxetex,ifluatex}
\usepackage{fixltx2e} % provides \textsubscript
\ifnum 0\ifxetex 1\fi\ifluatex 1\fi=0 % if pdftex
  \usepackage[T1]{fontenc}
  \usepackage[utf8]{inputenc}
\else % if luatex or xelatex
  \ifxetex
    \usepackage{mathspec}
  \else
    \usepackage{fontspec}
  \fi
  \defaultfontfeatures{Ligatures=TeX,Scale=MatchLowercase}
\fi
% use upquote if available, for straight quotes in verbatim environments
\IfFileExists{upquote.sty}{\usepackage{upquote}}{}
% use microtype if available
\IfFileExists{microtype.sty}{%
\usepackage{microtype}
\UseMicrotypeSet[protrusion]{basicmath} % disable protrusion for tt fonts
}{}
\usepackage[margin=1in]{geometry}
\usepackage{hyperref}
\hypersetup{unicode=true,
            pdftitle={Data Cleaning Project Read Me file},
            pdfauthor={Doug Joubert},
            pdfborder={0 0 0},
            breaklinks=true}
\urlstyle{same}  % don't use monospace font for urls
\usepackage{longtable,booktabs}
\usepackage{graphicx,grffile}
\makeatletter
\def\maxwidth{\ifdim\Gin@nat@width>\linewidth\linewidth\else\Gin@nat@width\fi}
\def\maxheight{\ifdim\Gin@nat@height>\textheight\textheight\else\Gin@nat@height\fi}
\makeatother
% Scale images if necessary, so that they will not overflow the page
% margins by default, and it is still possible to overwrite the defaults
% using explicit options in \includegraphics[width, height, ...]{}
\setkeys{Gin}{width=\maxwidth,height=\maxheight,keepaspectratio}
\IfFileExists{parskip.sty}{%
\usepackage{parskip}
}{% else
\setlength{\parindent}{0pt}
\setlength{\parskip}{6pt plus 2pt minus 1pt}
}
\setlength{\emergencystretch}{3em}  % prevent overfull lines
\providecommand{\tightlist}{%
  \setlength{\itemsep}{0pt}\setlength{\parskip}{0pt}}
\setcounter{secnumdepth}{0}
% Redefines (sub)paragraphs to behave more like sections
\ifx\paragraph\undefined\else
\let\oldparagraph\paragraph
\renewcommand{\paragraph}[1]{\oldparagraph{#1}\mbox{}}
\fi
\ifx\subparagraph\undefined\else
\let\oldsubparagraph\subparagraph
\renewcommand{\subparagraph}[1]{\oldsubparagraph{#1}\mbox{}}
\fi

%%% Use protect on footnotes to avoid problems with footnotes in titles
\let\rmarkdownfootnote\footnote%
\def\footnote{\protect\rmarkdownfootnote}

%%% Change title format to be more compact
\usepackage{titling}

% Create subtitle command for use in maketitle
\providecommand{\subtitle}[1]{
  \posttitle{
    \begin{center}\large#1\end{center}
    }
}

\setlength{\droptitle}{-2em}

  \title{Data Cleaning Project Read Me file}
    \pretitle{\vspace{\droptitle}\centering\huge}
  \posttitle{\par}
    \author{Doug Joubert}
    \preauthor{\centering\large\emph}
  \postauthor{\par}
      \predate{\centering\large\emph}
  \postdate{\par}
    \date{October 9, 2019}


\begin{document}
\maketitle

{
\setcounter{tocdepth}{2}
\tableofcontents
}
This DATACLEANINGPROJECT readme.txt file was generated on 2019-10-09 by
Douglas Joubert

\section{General Information}\label{general-information}

\begin{itemize}
\tightlist
\item
  This repository was created as part of the Getting and Cleaning Data
  course project. It has the instructions on how to run analysis on
  Human Activity recognition dataset.
\end{itemize}

\subsection{Author Information}\label{author-information}

Creator: Douglas Joubert

\subsection{Title of Dataset:}\label{title-of-dataset}

\subsubsection{Data set information}\label{data-set-information}

The experiments were carried out with a group of 30 volunteers within an
age bracket of 19-48 years. They performed a protocol of activities
composed of six basic activities: three static postures (standing,
sitting, lying) and three dynamic activities (walking, walking
downstairs and walking upstairs). The experiment also included postural
transitions that occurred between the static postures. These are:
stand-to-sit, sit-to-stand, sit-to-lie, lie-to-sit, stand-to-lie, and
lie-to-stand. All the participants were wearing a smartphone (Samsung
Galaxy S II) on the waist during the experiment execution. We captured
3-axial linear acceleration and 3-axial angular velocity at a constant
rate of 50Hz using the embedded accelerometer and gyroscope of the
device. The experiments were video-recorded to label the data manually.
The obtained dataset was randomly partitioned into two sets, where 70\%
of the volunteers was selected for generating the training data and 30\%
the test data {[}@Anguita{]}.

The sensor signals (accelerometer and gyroscope) were pre-processed by
applying noise filters and then sampled in fixed-width sliding windows
of 2.56 sec and 50\% overlap (128 readings/window). The sensor
acceleration signal, which has gravitational and body motion components,
was separated using a Butterworth low-pass filter into body acceleration
and gravity. The gravitational force is assumed to have only low
frequency components, therefore a filter with 0.3 Hz cutoff frequency
was used. From each window, a vector of features was obtained by
calculating variables from the time and frequency domain (Anguita,
2013).

Check the README.md for further details about this dataset.

\subsubsection{Date of data collection}\label{date-of-data-collection}

2012-12-10, more information available from Data Set Description
\href{http://archive.ics.uci.edu/ml/datasets/Human+Activity+Recognition+Using+Smartphones\#}{file}.

\subsubsection{Attribute Information:}\label{attribute-information}

For each record in the dataset it is provided: * Triaxial acceleration
from the accelerometer (total acceleration) and the estimated body
acceleration. * Triaxial Angular velocity from the gyroscope. * A
561-feature vector with time and frequency domain variables. * Its
activity label. * An identifier of the subject who carried out the
experiment.

\section{Code style and Data used}\label{code-style-and-data-used}

\begin{itemize}
\tightlist
\item
  This project was written in \emph{R version 3.5.1 (2018-07-02))}
\item
  Full description of the data used and analysis performed is found in
  the CodeBook.md or Codebook.rmd files.
\end{itemize}

\section{References}\label{references}

\begin{enumerate}
\def\labelenumi{\arabic{enumi}.}
\tightlist
\item
  Anguita, D., Ghio, A., Oneto, L., Parra, X., \& Reyes-Ortiz, J. L.
  (2013). Human Activity Recognition Using Smartphones Data Set.
  Retrieved October 4, 2019, from
  \url{http://archive.ics.uci.edu/ml/datasets/Human+Activity+Recognition+Using+Smartphones\#}
\end{enumerate}

\subsection{SHARING/ACCESS INFORMATION}\label{sharingaccess-information}

Licenses/restrictions placed on the data, or limitations of reuse:

Recommended citation for the data:

Citation for and links to publications that cite or use the data:

Links to other publicly accessible locations of the data:

Links/relationships to ancillary or related data sets:

\begin{longtable}[]{@{}l@{}}
\toprule
DATA \& FILE OVERVIEW\tabularnewline
\bottomrule
\end{longtable}

File list (filenames, directory structure (for zipped files) and brief
description of all data files):

Relationship between files, if important for context:

Additional related data collected that was not included in the current
data package:

If data was derived from another source, list source:

If there are there multiple versions of the dataset, list the file
updated, when and why update was made:

\begin{longtable}[]{@{}l@{}}
\toprule
METHODOLOGICAL INFORMATION\tabularnewline
\bottomrule
\end{longtable}

Description of methods used for collection/generation of data:

Methods for processing the data:

Software- or Instrument-specific information needed to interpret the
data, including software and hardware version numbers:

Standards and calibration information, if appropriate:

Environmental/experimental conditions:

Describe any quality-assurance procedures performed on the data:

People involved with sample collection, processing, analysis and/or
submission:

\begin{longtable}[]{@{}l@{}}
\toprule
DATA-SPECIFIC INFORMATION \tabularnewline
\bottomrule
\end{longtable}

Number of variables:

Number of cases/rows:

Variable list, defining any abbreviations, units of measure, codes or
symbols used:

Missing data codes:

Specialized formats or other abbreviations used:


\end{document}
